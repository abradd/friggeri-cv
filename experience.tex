Biostatistician, data analyst, and SAS developer in a broad range of applications. Specialize in organizing and analyzing large observational data sets. Extensive experience in statistical model building using SAS, including developing SAS macros and SAS/IML programs.\\

\section{Professional Experience}

\begin{entrylist}
  \entry
    {Since 2018}
    {Senior Biostatistician - Medical College of Wisconsin}
    {Milwaukee, WI}
    {
      \textit{Center for International Blood and Marrow Transplant Research}.
      % \vspace{-\baselineskip}
      \begin{itemize}
        \item Served as the Lead Biostatistician of the Cellular Therapy Function Area, responsible for overseeing and managing the study portfolio for both working committee and corporate studies.
        \item Provided statistical support and consultancy for multiple PASS studies of commercial CAR-T products in collaboration with pharmaceutical companies. Prepared SAPs, data summary reports, and annual statistical reports for FDA submissions.
        \item Oversaw the data retrieving process for the Cellular Therapy Data Registry. Ensured data integrity across multiple stages of data transition from Oracle database to SAS data extract.
        \item Provided supervision to entry-level statisticians, responsible for administrative functions such as hiring, mentoring, and managing workflows of statistical staffs.
      \end{itemize}
    }
  \entry
    {2011-2018}
    {Biostatistician - Medical College of Wisconsin}
    {Milwaukee, WI}
    {
      \textit{Center for International Blood and Marrow Transplant Research}.
      % \vspace{-\baselineskip}
      \begin{itemize}
        \item Conducted advanced statistical analyses for 2-3 research projects each year, including developing SAS and R programs, describing data, testing hypotheses, selecting models and interpreting results.
        \item Administered 5-10 research projects each year with close collaboration with principle investigators across the country. Ensured their successful publications on high impact scientific journals, such as \textit{New England Journal of Medicine}, \textit{Journal of Clinical Oncology}, and \textit{Blood}.
        \item Served as the developer and maintainer of the internal SAS code repository.
        \item Provided training and consultancy to entry-level statisticians.
      \end{itemize}
    }
  \entry
    {2009-2011}
    {Epidemiologist - Minnesota Department of Health}
    {St. Paul, MN}
    {
      \textit{Division of Health Promotion and Chronic Disease}.
      % \vspace{-\baselineskip}
      \begin{itemize}
        \item Conducted statistical analyses in SAS for the survey data of \textit{I Can Prevent Diabetes} program to evaluate its effectiveness.
        \item Prepared data for multiple scientific reports for successful grant applications and conference presentations.
      \end{itemize}
    }
  \entry
    {2006-2008}
    {Research Assistant - Fudan University}
    {Shanghai, China}
    {
      Performed data analyses in studies on striate cortex pattern adaptation of cats via intrinsic signal optical imaging method.
    }
\end{entrylist}

\section{Education}

\begin{entrylist}
%   \entry
%     {2018}
%     {Medical College of Wisconsin}
%     {Milwaukee, WI}
%     {Courseworks: applied survival analysis}
  \entry
    {2008-2010}
    {M.P.H. in Epidemiology - University of Minnesota, Twin Cities}
    {Minneapolis, MN}
    {Courseworks: SAS and R programming, generalized linear model, mixed model, survival analysis, factor analysis, epidemiological method, pathophysiology}
  \entry
    {2004–2008}
    {B.S. in Biological Science - Fudan University}
    {Shanghai, China}
    {Courseworks: biostatistics, bioinformatics, immunology, genetics, biochemistry}
\end{entrylist}
