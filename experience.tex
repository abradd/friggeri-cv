Biostatistician, data analyst, and SAS developer in a broad range of applications. Specialize in organizing and analyzing large observational data sets. Extensive experience in statistical model building using SAS, including developing SAS macros and SAS/IML programs.\\

\section{Professional Experience}

\begin{entrylist}
  \entry
    {since 2011}
    {Senior Biostatistician - Medical College of Wisconsin}
    {Milwaukee, WI}
    {
    \textit{Center for International Blood and Marrow Transplant Research}.
    % \vspace{-\baselineskip}
    \begin{itemize}
      \item Conducted advanced statistical analyses for 2-3 research projects each year, including developing SAS and R programs, describing data, testing hypotheses, selecting models and interpreting results.
      \item Administered 5-10 research projects each year, including developing protocols and analysis plans, reviewing manuscripts, and ensuring their successful publications on high impact scientific journals, such as \textit{New England Journal of Medicine}, \textit{Journal of Clinical Oncology}, \textit{Blood}, and \textit{Biology of Blood and Marrow Transplantation}.
      \item Provided statistical support and consultancy for collaborating studies with corporations and institutes, such as \textit{Novartis}, \textit{Kite Pharma}, and \textit{Magenta Therapeutics}.
      \item Oversaw the data quality control process for the CAR-T cellular therapy data registry, and ensured data integrity across multiple stages of data transition from Oracle databases to SAS data extracts.
      \item Supervised the development and maintainance of SAS code repository.
      \item Provided training and consultancy to entry-level statisticians.
    \end{itemize}
    }
  \entry
    {2009-2011}
    {Epidemiologist - Minnesota Department of Health}
    {St. Paul, MN}
    {
    \textit{Division of Health Promotion and Chronic Disease}.
    % \vspace{-\baselineskip}
    \begin{itemize}
      \item Provided statistical analyses in SAS for the survey data of \textit{I Can Prevent Diabetes} program to evaluate its efficacy.
      \item Prepared multiple scientific reports for successful grant applications and conference presentations.
    \end{itemize}
    }
  \entry
    {2006-2008}
    {Research Assistant - Fudan University}
    {Shanghai, China}
    {Provided data analyses in the investigations on striate cortex pattern adaptation of cats via intrinsic signal optical imaging method.}
\end{entrylist}

\section{Education}

\begin{entrylist}
%   \entry
%     {2018}
%     {Medical College of Wisconsin}
%     {Milwaukee, WI}
%     {Courseworks: applied survival analysis}
  \entry
    {2008-2010}
    {M.P.H. in Epidemiology - University of Minnesota, Twin Cities}
    {Minneapolis, MN}
    {Courseworks: SAS and R programming, generalized linear model, mixed model, survival analysis, factor analysis, epidemiological method, pathophysiology}
  \entry
    {2004–2008}
    {B.S. in Biological Science - Fudan University}
    {Shanghai, China}
    {Courseworks: biostatistics, bioinformatics, immunology, genetics, biochemistry}
\end{entrylist}
